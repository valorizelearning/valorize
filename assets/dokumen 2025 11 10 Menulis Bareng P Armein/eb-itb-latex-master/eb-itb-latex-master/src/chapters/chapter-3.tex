\chapter{ANALISIS DAN PERANCANGAN}
\label{chapter:3}

\section{pendahuluan}\label{pendahuluan}

Kedatangan era kecerdasan buatan (AI) telah mendorong pergeseran seismik
dalam praktik rekayasa dan pendidikan ilmu rekayasa. Perubahan ini
ditandai oleh \textbf{transformasi tak terduga} yang dihadapi oleh
profesi rekayasa.

\textbf{Implikasi Utama bagi Pendidikan dan Praktik Rekayasa:}

\begin{enumerate}
\def\labelenumi{\arabic{enumi}.}
\tightlist
\item
  \textbf{Ancaman Otomatisasi:} AI mengancam untuk mengotomatisasi
  kompetensi teknis yang selama ini ditekankan dalam pendidikan
  rekayasa,. Sebuah laporan dari World Economic Forum tahun 2023
  memperkirakan bahwa \textbf{47\% tugas rekayasa saat ini menghadapi
  otomatisasi dalam waktu lima tahun}.
\item
  \textbf{Kesenjangan Pedagogis:} Kondisi ini mengungkapkan kesenjangan
  pedagogis fundamental: pendidikan rekayasa memprioritaskan pengetahuan
  konten yang semakin banyak dilakukan oleh mesin, sambil
  \textbf{mengabaikan kemampuan khas manusia} yang sangat dibutuhkan
  industri, seperti penalaran etis dan pemecahan masalah yang
  kompleks,,.
\item
  \textbf{Pergeseran Fokus Nilai:} Tujuan belajar ilmu teknik tidak lagi
  hanya pada penguasaan materi (\emph{content mastery}),, tetapi
  bergeser secara mendalam menuju \textbf{``pembentukan sosok, karakter,
  dan pola berpikir profesi''},,,. Kompetensi profesional sejati kini
  terletak pada \emph{siapa} yang dibentuk dan \emph{bagaimana} ia
  berpikir, yang berfokus pada penilaian ahli dan pemecahan masalah
  adaptif.
\item
  \textbf{Kebutuhan Kompetensi Abad AI:} Diperlukan kerangka kerja
  (seperti CKMS-SE) yang mampu menumbuhkan kompetensi era AI---termasuk
  \textbf{kolaborasi, kreativitas, dan penalaran etis}---dengan
  memanfaatkan AI untuk memperkuat kekuatan manusia.
\end{enumerate}

\subsection{Valorise Learning}\label{valorise-learning}

Kerangka \textbf{VALORAIZE Learning} mewujudkan ``simulasi profesi''
sebagai strategi pedagogis intinya dengan secara fundamental mengubah
ruang kelas menjadi ekosistem ekonomi pengetahuan mikro (seperti
\textbf{Knowledge Marketplace}), di mana mahasiswa didorong untuk
bertindak sebagai \textbf{produsen nilai} dan \textbf{rekayasawan
profesional},,.

Simulasi ini diwujudkan di ruang kelas melalui integrasi sistem
\textbf{Knowledge Marketplace} dan penciptaan \textbf{artefak
pengetahuan otentik} melalui mekanisme berikut:

\subsection{1. Transformasi Peran dan Filosofi (Simulasi
Profesi)}\label{transformasi-peran-dan-filosofi-simulasi-profesi}

\begin{itemize}
\tightlist
\item
  \textbf{Pergeseran Filosofis:} Fokus pendidikan bergeser dari sekadar
  penguasaan materi (\emph{content mastery}) menjadi \textbf{pembentukan
  sosok, karakter, dan pola berpikir profesi},. Tujuannya adalah
  membimbing mahasiswa untuk \textbf{berpikir dan bertindak layaknya
  seorang insinyur profesional},.
\item
  \textbf{Peran Dosen:} Dosen bertransisi dari sekadar pemberi informasi
  menjadi \textbf{representasi dan teladan dari profesi} tersebut,, yang
  bertujuan menumbuhkan \textbf{identitas profesional} mahasiswa,.
\item
  \textbf{Mengatasi Kesenjangan:} Simulasi ini dirancang untuk
  menjembatani kesenjangan antara teori dan praktik, serta menumbuhkan
  kemampuan pemecahan masalah kompleks dan pengambilan keputusan etis
  yang tidak mudah diotomatisasi oleh Kecerdasan Buatan (AI),.
\end{itemize}

\subsection{2. Penciptaan Artefak Pengetahuan yang Personal dan
Otentik}\label{penciptaan-artefak-pengetahuan-yang-personal-dan-otentik}

Artefak adalah bukti nyata dari pemahaman mendalam dan merupakan
\textbf{produk pengetahuan} yang bernilai tinggi,.

\begin{itemize}
\tightlist
\item
  \textbf{Peta Pengetahuan:} Bentuk inti dari artefak ini adalah
  \textbf{Peta Pengetahuan} (Knowledge Maps),,.

  \begin{itemize}
  \tightlist
  \item
    \textbf{Peta Pengetahuan Primitif:} Berfungsi sebagai ``badan
    pengetahuan'' inti yang merepresentasikan konsep-konsep fundamental
    dan hubungan dasarnya (pengetahuan deklaratif),.
  \item
    \textbf{Peta Pemecahan Masalah:} Bersifat dinamis dan berorientasi
    proses. Mahasiswa harus mengkonseptualisasikan masalah sebagai
    ``\textbf{celah}'' antara titik mulai dan titik akhir, kemudian
    memetakan ``\textbf{rute}'' dan memilih ``\textbf{kendaraan}''
    (alat, teknik, algoritma, heuristik) yang sesuai untuk solusinya,.
    Ini meniru pemikiran strategis layaknya ahli,.
  \end{itemize}
\item
  \textbf{Nilai Komunitas:} Artefak yang dihasilkan oleh mahasiswa ini
  merupakan \textbf{kontribusi nyata terhadap basis pengetahuan
  bersama},, yang berfungsi sebagai aset komunitas dan bukan sekadar
  tugas yang dinilai.
\end{itemize}

\subsection{3. Knowledge Marketplace (Ekonomi Penilaian
Dinamis)}\label{knowledge-marketplace-ekonomi-penilaian-dinamis}

Knowledge Marketplace adalah sistem asesmen yang mengubah dinamika
penilaian menjadi proses \textbf{Penciptaan Nilai Bersama (Value
Co-Creation/VCC)}, menyerupai pertukaran profesional,.

\begin{itemize}
\tightlist
\item
  \textbf{Dosen Menciptakan Permintaan (Demand Creator):} Setiap minggu,
  dosen ``\textbf{mengiklankan}'' kebutuhan akan ``karya pengetahuan dan
  pemecahan masalah'' tertentu, menargetkan topik dan tingkat
  \textbf{Taksonomi Bloom} spesifik,,.
\item
  \textbf{Mahasiswa Menciptakan Nilai (Value Creator):} Mahasiswa
  merespons iklan ini dengan menghasilkan artefak (Peta Pengetahuan),.
  Tindakan sukarela dalam menciptakan artefak ini memenuhi kebutuhan
  \textbf{otonomi} mahasiswa, yang selaras dengan Teori Penentuan Diri
  (SDT),.
\item
  \textbf{Transaksi dan Insentif:} Dosen ``\textbf{membeli}'' karya
  berkualitas tinggi menggunakan sistem \textbf{mata uang digital
  berjenjang} yang diselaraskan dengan tingkat kognitif:

  \begin{itemize}
  \tightlist
  \item
    \textbf{Point Uang} (Level 1-2 Bloom: Mengingat dan Memahami).
  \item
    \textbf{Point Emas} (Level 3 Bloom: Menerapkan).
  \item
    \textbf{Point Platinum} (Level 4-5 Bloom: Menganalisis dan
    Mengevaluasi).
  \item
    \textbf{Point Berlian} (Level 6 Bloom: Menciptakan).
  \end{itemize}
\item
  \textbf{Publikasi dan Kontribusi Komunitas:} Karya yang telah dibeli
  akan \textbf{diunggah ke situs web kuliah}, menjadikannya
  \textbf{sumber belajar yang berharga bagi mahasiswa di tahun
  berikutnya},. Ini menumbuhkan \textbf{rasa menciptakan nilai} dan
  kepemilikan kolektif,, dan mengubah penilaian dari sekadar ``penjagaan
  gerbang institusional'' menjadi \textbf{kontribusi pengetahuan
  autentik} kepada komunitas,.
\end{itemize}

\subsection{CKM-SE}\label{ckm-se}

CKMS-SE adalah \textbf{Collaborative Knowledge Management System-Smart
Engineering}. CKMS-SE diusulkan sebagai \textbf{Smart Artefact} untuk
transformasi pendidikan tinggi yang berorientasi nilai.

\textbf{1. Apa itu CKMS-SE?}

CKMS-SE adalah Kerangka Sistem Manajemen Pengetahuan Kolaboratif Cerdas
berbasis Rekayasa Cerdas (\emph{Smart Engineering}) yang dirancang untuk
\textbf{Pendidikan Tinggi Berorientasi Nilai (Value-Oriented
Education/VOE)}.

Arsitektur CKMS-SE mengimplementasikan paradigma
\textbf{Triune-Intelligence Smart Engineering (TISE)} yang menyinergikan
tiga jenis kecerdasan: kecerdasan manusia (\emph{Homocordium}),
kecerdasan buatan (\emph{Homologos}), dan kecerdasan alamiah.

Framework ini terdiri dari tiga komponen inti yang terintegrasi: *
\textbf{Core Engine:} Platform CKMS fundamental. * \textbf{PUDAL
Engine:} Mesin kognitif berbasis Kecerdasan Buatan (AI) yang mendukung
\textbf{Pembelajaran Personal (PL) dinamis} dan memfasilitasi
\textbf{Pembelajaran Kolaboratif (CL)} yang efektif. * \textbf{PSKVE
Engine:} Mesin manajemen nilai yang berfungsi mengelola dan
mengoptimalkan \textbf{Penciptaan Nilai Bersama (Value Co-Creation/VCC)}
dalam konteks VOE.

\textbf{2. Mengapa CKMS-SE Diusulkan?}

CKMS-SE diusulkan sebagai inisiatif strategis untuk menciptakan
pengalaman belajar yang bermakna dan untuk melakukan pergeseran
paradigma pendidikan:

\begin{itemize}
\tightlist
\item
  \textbf{Mengatasi Keterbatasan Tradisional:} Pendekatan rekayasa
  sistem pendidikan tradisional sering menghadapi keterbatasan dalam
  mengatasi kompleksitas dan tidak memfasilitasi VCC secara optimal.
\item
  \textbf{Mengalihkan Fokus dari Materi ke Identitas:} Tujuan belajar
  ilmu teknik tidak lagi hanya pada penguasaan pengetahuan teknis,
  tetapi pada \textbf{pembentukan sosok, karakter, dan pola berpikir
  profesi}, mempersiapkan lulusan yang mampu menghasilkan solusi yang
  bernilai tinggi.
\item
  \textbf{Menjembatani Kesenjangan (Theory-Practice Gap):} CKMS-SE
  adalah respons strategis terhadap kesenjangan yang diamati antara
  lulusan yang secara teknis kompeten dan tuntutan industri yang
  memerlukan kemampuan pemecahan masalah kompleks, penalaran etis, dan
  kolaborasi.
\end{itemize}

\textbf{3. Masalah Apa yang Hendak Diatasi?}

CKMS-SE secara khusus hendak mengatasi beberapa kesenjangan utama dalam
sistem pembelajaran saat ini:

\begin{itemize}
\tightlist
\item
  \textbf{Model Pembelajaran Konseptual:} Model yang ada masih bersifat
  konseptual dan belum diuji dalam situasi nyata, dengan terbatasnya
  panduan implementasi detail untuk VCC.
\item
  \textbf{Dukungan Konten Mahasiswa:} Sistem pembelajaran yang ada belum
  secara spesifik mendukung konten pembelajaran yang dibuat oleh
  mahasiswa (\emph{student-generated content}).
\item
  \textbf{Evaluasi Interaksi AI:} Ketiadaan mekanisme evaluasi yang
  komprehensif untuk mengukur efektivitas interaksi mahasiswa dengan AI.
\item
  \textbf{Kesenjangan Kompetensi Era AI:} Mengatasi masalah bahwa
  pendidikan teknik konvensional mengutamakan kompetensi teknis yang
  rentan diotomatisasi oleh AI, sementara \textbf{kemampuan khas
  manusia} seperti kreativitas, penalaran etis, dan kolaborasi
  diabaikan.
\end{itemize}

\section{TISE dAn Penerapnya}\label{tise-dan-penerapnya}

\textbf{TISE} adalah singkatan dari \textbf{Triune-Intelligence Smart
Engineering}, sebuah paradigma arsitektural yang diimplementasikan oleh
CKMS-SE.

TISE dirancang untuk mengintegrasikan dan menyinergikan tiga jenis
kecerdasan secara seimbang:

\begin{enumerate}
\def\labelenumi{\arabic{enumi}.}
\tightlist
\item
  \textbf{Kecerdasan Manusia} (\emph{Homocordium})
\item
  \textbf{Kecerdasan Buatan} (\emph{Homologos})
\item
  \textbf{Kecerdasan Alamiah} (atau Kecerdasan Kultural/Lingkungan).
\end{enumerate}

Tujuan dari implementasi TISE adalah untuk memastikan sistem
pembelajaran cerdas yang dihasilkan tidak hanya \textbf{cerdas secara
teknis} tetapi juga \textbf{humanis dan berorientasi nilai}.

\subsection{Mekanisme TISE dalam
CKMS-SE}\label{mekanisme-tise-dalam-ckms-se}

TISE memastikan keseimbangan dan penyelarasan nilai antara ketiga
kecerdasan tersebut dalam CKMS-SE melalui perancangan arsitektur dan
siklus operasional:

\begin{enumerate}
\def\labelenumi{\arabic{enumi}.}
\tightlist
\item
  \textbf{Penyelarasan Nilai (Value Alignment):} TISE dirancang untuk
  menghasilkan tingkat \textbf{penyelarasan nilai yang lebih tinggi}
  antara ketiga kecerdasan. Hal ini memastikan artefak rekayasa selaras
  dengan \textbf{nilai-nilai manusia} dan konteks budaya/lingkungan.
\item
  \textbf{Kerangka Kolaboratif Tripartit:} Dalam perancangan Lapisan
  Sistem CKMS-SE, TISE menentukan bagaimana \textbf{Kecerdasan
  Tripartit} (NI, CI, AI) akan berkolaborasi untuk menciptakan sistem
  pembelajaran yang \textbf{holistik dan seimbang}.
\item
  \textbf{Siklus Kognitif PUDAL:} Keseimbangan ini dioperasionalkan
  secara berkelanjutan dalam siklus kognitif PUDAL
  (Perceive-Understand-Decision-Act-Learning-Evaluate):

  \begin{itemize}
  \tightlist
  \item
    Meskipun Kecerdasan Buatan (AI) dan Kecerdasan Kultural/Alamiah (CI)
    membantu dalam fase \emph{Perceive} dan \emph{Understand},
    \textbf{Keputusan utama (WHAT)} dalam siklus pembelajaran
    \textbf{selalu dipegang oleh Kecerdasan Manusia (NI)}.
  \end{itemize}
\item
  \textbf{Inovasi Holistik:} TISE menggeser fokus rekayasa dari
  pemecahan masalah teknis murni menjadi \textbf{penciptaan ``teater
  kehidupan yang megah''} bagi kemanusiaan dan lingkungan.
\end{enumerate}

\subsection{implementasi tISE asa
CKMS-SE}\label{implementasi-tise-asa-ckms-se}

\textasciitilde\textasciitilde{} CKMS-SE (Collaborative Knowledge
Management System-Smart Engineering) mengimplementasikan arsitektur
\textbf{Triune-Intelligence Smart Engineering (TISE)} yang menyinergikan
tiga jenis kecerdasan (\emph{Homocordium}, \emph{Homologos}, dan
kecerdasan alamiah) untuk memastikan \textbf{penyelarasan nilai} dalam
sistem pembelajaran cerdas.

TISE berinteraksi dengan \textbf{PSKVE Engine} dengan menjadikan PSKVE
sebagai mesin manajemen nilai yang \textbf{mengoperasionalkan Penciptaan
Nilai Bersama (VCC)} dalam konteks Pendidikan Berorientasi Nilai (VOE).
PSKVE Engine mengelola dan mengoptimalkan VCC dalam lima dimensi energi,
memastikan bahwa nilai yang dihasilkan oleh kolaborasi
manusia-AI-alamiah bersifat holistik dan terukur.

Berikut adalah bagaimana PSKVE Engine mengelola VCC dalam lima dimensi
energi tersebut:

\begin{enumerate}
\def\labelenumi{\arabic{enumi}.}
\tightlist
\item
  \textbf{Product Energy (PE):} Mengelola kualitas dan utilitas
  \emph{platform} CKMS yang dihasilkan. \textbf{VCC terjadi} ketika
  mahasiswa memberikan \textbf{umpan balik untuk perbaikan
  \emph{platform}}, misalnya melaporkan \emph{bug} atau berpartisipasi
  dalam pengembangan fitur baru.
\item
  \textbf{Service Energy (SE):} Mengelola kualitas layanan dan dukungan
  yang diberikan sistem. \textbf{VCC terjadi} ketika CKMS menciptakan
  nilai layanan melalui umpan balik AI yang instan dan dipersonalisasi,
  dan ketika \textbf{mahasiswa saling membantu} dalam forum untuk
  membentuk budaya dukungan sejawat.
\item
  \textbf{Knowledge Energy (KE):} Mengelola kapasitas sistem dalam
  memfasilitasi penciptaan (\emph{creation}) dan pembagian
  (\emph{sharing}) pengetahuan. \textbf{VCC terjadi} sebagai proses di
  mana pengetahuan ini disempurnakan secara kolaboratif melalui
  komentar, perbaikan, dan pengembangan berkelanjutan.
\item
  \textbf{Value Energy (VE):} Mengelola dampak ekonomi dan sosial yang
  dihasilkan sistem. \textbf{VCC terjadi} ketika kolaborasi dalam CKMS
  menghasilkan \emph{output} yang nilainya \textbf{lebih besar daripada
  jumlah kontribusi individu} (misalnya, nilai ekonomi, sosial, atau
  reputasi).
\item
  \textbf{Environmental Energy (EE):} Mengelola keberlanjutan dan dampak
  lingkungan sistem. Nilai diciptakan dengan merancang CKMS yang
  mendorong \textbf{interaksi positif dan etis}, melindungi privasi
  data, dan mempromosikan inklusivitas.
\end{enumerate}

Secara ringkas, TISE memberikan prinsip desain filosofis yang menuntut
penyelarasan kecerdasan, sementara PSKVE Engine adalah mekanisme
rekayasa yang spesifik untuk mengukur dan mengelola hasil dari
penyelarasan nilai tersebut dalam lima kategori terukur.

\section{Research Question dan
Metode}\label{research-question-dan-metode}

Pertanyaan penelitian utama (\emph{research questions} atau RQ) yang
mendasari pengembangan dan eksperimen CKMS-SE (Collaborative Knowledge
Management System-Smart Engineering) terbagi menjadi pertanyaan yang
lebih luas yang memandu perancangan \emph{framework}, dan pertanyaan
spesifik yang memandu studi kuasi-eksperimental tentang dampaknya.

\subsection{\texorpdfstring{Pertanyaan Penelitian Proyek (Tujuan
Pengembangan \emph{Framework}
CKMS-SE):}{Pertanyaan Penelitian Proyek (Tujuan Pengembangan Framework CKMS-SE):}}\label{pertanyaan-penelitian-proyek-tujuan-pengembangan-framework-ckms-se}

Secara umum, penelitian ini berupaya menjawab tiga pertanyaan utama:

\begin{enumerate}
\def\labelenumi{\arabic{enumi}.}
\tightlist
\item
  \textbf{Bagaimana mengembangkan \emph{framework} berbasis AI} di
  pendidikan tinggi yang mengintegrasikan CKMS, Pembelajaran Kolaboratif
  (CL), dan Pembelajaran Personal (PL), untuk mendukung Pendidikan
  Berorientasi Nilai (VOE), dan Penciptaan Nilai Bersama (VCC)?
\item
  \textbf{Bagaimana merancang metrik evaluasi} untuk menilai
  efektivitas, keterlibatan, dan keberhasilan dari \emph{framework}
  pembelajaran tersebut secara valid dan reliabel?
\item
  \textbf{Bagaimana mengukur efektivitas implementasi} sistem
  pembelajaran berbasis AI, CKMS, CL, PL dalam meningkatkan keterlibatan
  belajar, pencapaian nilai, dan proses \emph{co-creation} di pendidikan
  tinggi?
\end{enumerate}

\subsection{Pertanyaan Penelitian Eksperimental
Spesifik:}\label{pertanyaan-penelitian-eksperimental-spesifik}

Pertanyaan penelitian yang memandu studi kuasi-eksperimental yang
mengukur hasil kuantitatif (motivasi, identitas profesional, transfer
pembelajaran) adalah:

\begin{itemize}
\tightlist
\item
  \textbf{Apakah sistem manajemen pengetahuan kolaboratif yang dirancang
  melalui prinsip penciptaan nilai bersama (VCC) meningkatkan motivasi
  intrinsik dan pembentukan identitas profesional dalam pendidikan
  rekayasa, sekaligus menghasilkan nilai komunitas yang terukur melalui
  penggunaan kembali artefak?}
\end{itemize}

\section{eksperimen danhASIL}\label{eksperimen-danhasil}

Eksperimen yang dilakukan adalah \textbf{studi kuasi-eksperimental} yang
membandingkan kelompok yang menerima intervensi CKMS-SE dengan kelompok
kontrol yang menerima instruksi konvensional.

\textbf{Intervensi dan Populasi:}

\begin{itemize}
\tightlist
\item
  \textbf{Intervensi (Intervention/I):} Implementasi \textbf{CKMS-SE
  (Collaborative Knowledge Management System-Smart Engineering)}, yang
  didasarkan pada Teori Penentuan Diri (Self-Determination Theory/SDT)
  dan Teori Beban Kognitif (Cognitive Load Theory/CLT).
\item
  \textbf{Komponen Intervensi:} CKMS-SE terdiri dari tiga komponen
  terintegrasi: \textbf{Knowledge Marketplace} (untuk penciptaan artefak
  sukarela), \textbf{Knowledge Maps} (alat visualisasi berbasis AI untuk
  eksternalisasi skema konseptual), dan \textbf{Socratic AI Coaching}
  (untuk umpan balik real-time yang dipersonalisasi).
\item
  \textbf{Populasi (Population/P):} 142 mahasiswa sarjana teknik tahun
  ketiga (kelompok intervensi n=71, kelompok kontrol n=71).
\item
  \textbf{Konteks (Context/Cx):} Dilakukan selama satu semester (12
  minggu) di Institut Teknologi Bandung pada dua mata kuliah inti
  paralel (Probabilitas \& Statistik dan Rekayasa Sistem).
\end{itemize}

\textbf{Hasil Penelitian (Outcomes/O):}

Hasil kuantitatif menunjukkan \textbf{dampak pedagogis yang besar} pada
kelompok intervensi di tiga metrik utama, yang diukur setelah intervensi
12 minggu:

\begin{longtable}[]{@{}
  >{\raggedright\arraybackslash}p{(\linewidth - 6\tabcolsep) * \real{0.2500}}
  >{\raggedright\arraybackslash}p{(\linewidth - 6\tabcolsep) * \real{0.2500}}
  >{\raggedright\arraybackslash}p{(\linewidth - 6\tabcolsep) * \real{0.2500}}
  >{\raggedright\arraybackslash}p{(\linewidth - 6\tabcolsep) * \real{0.2500}}@{}}
\toprule\noalign{}
\begin{minipage}[b]{\linewidth}\raggedright
Pengukuran
\end{minipage} & \begin{minipage}[b]{\linewidth}\raggedright
Kelompok Kontrol (Rata-rata/SD)
\end{minipage} & \begin{minipage}[b]{\linewidth}\raggedright
Kelompok Intervensi (Rata-rata/SD)
\end{minipage} & \begin{minipage}[b]{\linewidth}\raggedright
Ukuran Efek (Cohen's d)
\end{minipage} \\
\midrule\noalign{}
\endhead
\bottomrule\noalign{}
\endlastfoot
\textbf{Motivasi Intrinsik} & 3.2 (0.8) & 4.1 (0.6) & \textbf{0.97} \\
\textbf{Identitas Profesional} & 3.4 (0.7) & 4.2 (0.7) &
\textbf{0.84} \\
\textbf{Transfer Pembelajaran} & 2.9 (0.9) & 4.1 (0.8) &
\textbf{0.95} \\
\end{longtable}

\textbf{1. Hasil Kuantitatif Utama:} * \textbf{Motivasi Intrinsik:}
Peningkatan signifikan dengan ukuran efek besar (\emph{d} = 0.97, p
\textless{} .001). Ini didorong oleh \textbf{otonomi} yang diberikan
melalui penciptaan artefak secara sukarela. * \textbf{Identitas
Profesional:} Peningkatan substansial (\emph{d} = 0.84), menunjukkan
pergeseran persepsi diri mahasiswa dari ``konsumen pengetahuan pasif''
menjadi ``\textbf{pencipta pengetahuan dan kontributor komunitas}''. *
\textbf{Transfer Pembelajaran:} Peningkatan signifikan (\emph{d} =
0.95), menunjukkan mahasiswa mampu menerapkan strategi penalaran yang
dipelajari pada konteks pemecahan masalah yang belum diajarkan (novel
problem-solving untaught in course).

\textbf{2. Metrik Platform dan Penciptaan Nilai Komunitas:} *
\textbf{Penciptaan Artefak:} Mahasiswa menghasilkan \textbf{427 artefak}
yang ditinjau sejawat. * \textbf{Tinjauan Sejawat:} Artefak tersebut
menerima \textbf{1.284 tinjauan sejawat} (rata-rata 9.2 tinjauan per
artefak), menunjukkan keterlibatan komunitas yang kuat. *
\textbf{Realisasi Nilai Bersama (VCC):} \textbf{16\% dari artefak (n=68)
digunakan kembali} oleh kohort semester berikutnya. Artefak yang paling
sering digunakan kembali (lebih dari 10 kali) membahas topik yang
menantang secara konseptual, menunjukkan bahwa penjelasan yang dibuat
oleh rekan sejawat lebih mudah diakses daripada buku teks formal.

\textbf{3. Hasil Kualitatif (Mekanisme VCC):} Analisis tematik
mengidentifikasi tiga mekanisme nilai bersama yang menjelaskan
keberhasilan CKMS-SE: * \textbf{Nilai Epistemologis:} Proses pembuatan
artefak memaksa mahasiswa menghadapi pengetahuan yang tidak lengkap dan
membangun \textbf{model mental yang lebih kuat}. * \textbf{Nilai
Sosial:} Akuntabilitas tinjauan sejawat mendorong produksi artefak
berkualitas lebih tinggi karena mahasiswa tahu ``orang sungguhan akan
menggunakan ini'' (real people will use this). * \textbf{Nilai
Identitas:} Mahasiswa melaporkan status ``\textbf{pakar junior}''
melalui kontribusi artefak dan pengakuan komunitas.

\begin{center}\rule{0.5\linewidth}{0.5pt}\end{center}

Penelitian ini menunjukkan bahwa CKMS-SE berhasil menumbuhkan kompetensi
yang dibutuhkan di era AI, seperti penalaran etis, kreativitas, dan
kolaborasi. Apakah Anda ingin mendalami bagaimana TISE
(Triune-Intelligence Smart Engineering) memastikan bahwa kecerdasan
buatan, manusia, dan alamiah diseimbangkan dalam proses pembelajaran
CKMS-SE?

\section{eiskux=zi}\label{eiskuxzi}

Tiga komponen inti dari CKMS-SE (Collaborative Knowledge Management
System-Smart Engineering)---yaitu \textbf{Knowledge Marketplace,
Knowledge Maps, dan Socratic AI Coaching}---bekerja sama secara sinergis
untuk mencapai hasil peningkatan motivasi, identitas profesional, dan
transfer pembelajaran yang signifikan.

Kerja sama holistik ini didasarkan pada kerangka kerja Penciptaan Nilai
Bersama (VCC) yang berakar pada \textbf{Teori Penentuan Diri
(Self-Determination Theory/SDT)} dan \textbf{Teori Beban Kognitif
(Cognitive Load Theory/CLT)}.

Berikut adalah mekanisme kerja sama dari ketiga komponen tersebut:

\begin{enumerate}
\def\labelenumi{\arabic{enumi}.}
\tightlist
\item
  \textbf{Knowledge Marketplace (Meningkatkan Motivasi Intrinsik dan
  Identitas Profesional):}

  \begin{itemize}
  \tightlist
  \item
    \textbf{Mekanisme:} \emph{Marketplace} menyediakan kerangka kerja di
    mana mahasiswa bertindak sebagai ``\textbf{knowledge producer}''
    (pencipta pengetahuan) daripada konsumen pasif.
  \item
    \textbf{Fungsi:} Komponen ini memenuhi kebutuhan psikologis akan
    \textbf{otonomi} (Autonomy) dalam SDT, karena mahasiswa secara
    \textbf{sukarela} merespons ``kebutuhan belajar'' yang diiklankan
    oleh instruktur dengan membuat artefak (video, peta konsep,
    simulasi).
  \item
    \textbf{Kolaborasi:} Sistem ini mengintegrasikan \textbf{Peer
    Review} (penilaian sejawat), yang mengubah penilaian dari sekadar
    ``penjagaan gerbang institusional'' menjadi \textbf{kontribusi
    pengetahuan yang autentik} kepada komunitas, sehingga memenuhi
    kebutuhan akan \textbf{keterhubungan} (Relatedness).
  \item
    \textbf{Hasil:} Sistem mata uang berjenjang (Point Uang, Emas,
    Platinum, Berlian) yang diselaraskan dengan Taksonomi Bloom
    memberikan umpan balik yang terdiferensiasi dan memotivasi progresi
    kognitif.
  \end{itemize}
\item
  \textbf{Knowledge Maps (Meningkatkan Transfer Pembelajaran dan
  Kompetensi):}

  \begin{itemize}
  \tightlist
  \item
    \textbf{Mekanisme:} Komponen ini menyediakan alat visualisasi
    berbantuan AI yang memungkinkan mahasiswa untuk
    \textbf{mengeksternalisasi skema konseptual} mereka.
  \item
    \textbf{Fungsi:} Sesuai dengan CLT, Peta Pengetahuan mengurangi
    \textbf{beban kognitif ekstrinsik} dengan membuat pengetahuan ahli
    yang implisit menjadi eksplisit dan mudah dimanipulasi.
  \item
    \textbf{Kolaborasi:} Proses pembuatan artefak (terutama Peta
    Pengetahuan Primitif dan Peta Pemecahan Masalah) memaksa mahasiswa
    untuk menyusun model mental yang lebih kuat. Artefak yang paling
    sering digunakan kembali oleh kohort berikutnya adalah yang membahas
    topik konseptual yang menantang, menunjukkan bahwa penjelasan yang
    dibuat oleh rekan sejawat (peer-created explanations) lebih mudah
    diakses daripada buku teks formal.
  \end{itemize}
\item
  \textbf{Socratic AI Coaching (Memperkuat Kompetensi dan Otonomi):}

  \begin{itemize}
  \tightlist
  \item
    \textbf{Mekanisme:} Ini adalah integrasi API yang memberikan
    \textbf{umpan balik personal real-time} pada draf artefak.
  \item
    \textbf{Fungsi:} AI Coaching dirancang untuk memberikan
    \textbf{scaffolding \emph{just-in-time}} (bantuan tepat waktu) dalam
    Zona Perkembangan Proksimal (Zone of Proximal Development)
    pembelajar. Alih-alih memberikan jawaban langsung, sistem ini
    mengajukan \textbf{pertanyaan klarifikasi strategis} dan
    mengidentifikasi kesenjangan pengetahuan.
  \item
    \textbf{Integrasi:} Pendekatan ini memungkinkan pemecahan masalah
    secara mandiri (menghormati \textbf{otonomi} mahasiswa) sambil
    memberikan struktur dukungan yang diperlukan (meningkatkan
    \textbf{kompetensi}).
  \end{itemize}
\end{enumerate}

\textbf{Peran Mesin Penggerak (PUDAL dan PSKVE):} CKMS-SE
mengimplementasikan siklus kognitif berkelanjutan melalui \textbf{PUDAL
Engine} (Perceive-Understand-Decision-Act-Learning-Evaluate) yang
membuat sistem menjadi adaptif terhadap kebutuhan pembelajar. Sementara
itu, \textbf{PSKVE Engine}
(Product-Service-Knowledge-Value-Environmental) memastikan bahwa nilai
yang diciptakan (seperti tingkat penggunaan kembali artefak sebesar
16\%) bersifat holistik dan merupakan aset komunitas yang asli.

\section{kesimpulan}\label{kesimpulan}

Berdasarkan hasil yang dicapai, kesimpulan utama yang dapat dilaporkan
berpusat pada efektivitas dan arsitektur dari \textbf{CKMS-SE
(Collaborative Knowledge Management System-Smart Engineering)}.

Kesimpulan penelitian CKMS-SE/VALORAIZE menunjukkan bahwa:

\begin{enumerate}
\def\labelenumi{\arabic{enumi}.}
\tightlist
\item
  \textbf{CKMS-SE Framework} yang diusulkan didasarkan pada paradigma
  \textbf{Smart Engineering} dan mengintegrasikan teknologi cerdas dan
  pembelajaran kolaboratif. Framework ini terdiri dari tiga komponen
  utama: \textbf{Core Engine} (untuk pembelajaran kolaboratif),
  \textbf{PUDAL Engine} berbasis AI (untuk pembelajaran personal), dan
  \textbf{PSKVE Engine} (untuk optimalisasi Value Co-Creation/VCC).
\item
  Framework ini dirancang untuk secara signifikan lebih efektif dalam
  mendukung \textbf{Pendidikan Berorientasi Nilai (VOE)} dan
  memfasilitasi \textbf{Penciptaan Nilai Bersama (VCC)} dibandingkan
  dengan pendekatan konvensional atau sistem yang kurang terintegrasi.
\item
  Implementasi CKMS-SE (yang didukung oleh Teori Penentuan Diri dan
  Teori Beban Kognitif) telah terbukti berhasil membudidayakan
  kompetensi yang dibutuhkan di era AI---seperti kolaborasi,
  kreativitas, dan penalaran etis---dengan memanfaatkan kecerdasan
  buatan untuk memperkuat kekuatan manusia.
\item
  Hasil kuantitatif menunjukkan \textbf{dampak pedagogis yang besar}
  pada kelompok intervensi, termasuk peningkatan signifikan pada
  \textbf{motivasi intrinsik} (Cohen's d=0.97), \textbf{identitas
  profesional} (d=0.84), dan \textbf{transfer pembelajaran} (d=0.95).
\item
  Sistem ini menghasilkan \textbf{penciptaan nilai komunitas otentik}
  yang terukur, dibuktikan dengan tingkat penggunaan kembali artefak
  pengetahuan (seperti Peta Pengetahuan) yang dibuat oleh mahasiswa
  (16\% artefak digunakan kembali oleh angkatan semester berikutnya).
\end{enumerate}

Secara keseluruhan, penelitian menyimpulkan bahwa CKMS-SE menawarkan
jalur yang menarik untuk mempersiapkan lulusan yang inovatif dan
bertanggung jawab dalam sistem teknis dan sosial yang kompleks.

\section{Referensi}\label{referensi}


\section{Analisis Masalah}
\blindtext

\section{Solusi Umum}
\blindtext

\section{Rancangan Solusi}
\blindtext